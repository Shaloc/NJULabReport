%---------------------------------------------------------------------
%
%                             NJU LabReport 
%			A XeLaTeX Template modified from NUST LabReport
%
%---------------------------------------------------------------------
%               
%			 	  modified by Shaloc <shalocn@outlook.com>
%		created by and thanks to Qingyun Fang <fqy2017@gmail.com>
%					https://github.com/DocF/NjustLabReport
%                        Last-modified: 2018-10-05
%
%---------------------------------------------------------------------

\documentclass[a4paper,12pt]{report}
\usepackage{ctex}
\usepackage{times}
\usepackage{setspace}
\usepackage{fancyhdr}
\usepackage{graphicx}
\usepackage{wrapfig}
\usepackage{array}  
\usepackage{fontspec,xunicode,xltxtra}
\usepackage{titlesec}
\usepackage{titletoc}
\usepackage[titletoc]{appendix}
\usepackage[top=30mm,bottom=30mm,left=20mm,right=20mm]{geometry}
\usepackage{cite}
\usepackage{listings}
\usepackage[framed,numbered,autolinebreaks,useliterate]{mcode} % 插入代码
\usepackage{minted} 										   % 代码高亮
\usepackage{amsmath} 										   % 插入定理
\usepackage{booktabs} 										   % 插入三线表格
\usepackage{fancyvrb}
\XeTeXlinebreaklocale "zh"
\XeTeXlinebreakskip = 0pt plus 1pt minus 0.1pt
%代码字体 monaco
\newcommand{\monaco}{\textit{\textbf{\textsf{Monaco}}}}

%---------------------------------------------------------------------
%	页眉页脚设置, 如果不需要这部分可以使用\iffalse \fi注释掉
%---------------------------------------------------------------------
\fancypagestyle{plain}{
	\pagestyle{fancy}      %改变章节首页页眉
}

\pagestyle{fancy}
\lhead{\kaishu~文献综述~}
\rhead{}
\cfoot{\thepage}

%---------------------------------------------------------------------
%	章节标题设置
%---------------------------------------------------------------------
\titleformat{\chapter}{\centering\zihao{-1}\heiti}{}{1em}{}
\titlespacing{\chapter}{0pt}{*0}{*6}

%---------------------------------------------------------------------
%	摘要标题设置
%---------------------------------------------------------------------
\renewcommand{\abstractname}{\zihao{-3} 摘\quad 要}

%---------------------------------------------------------------------
%	参考文献设置
%---------------------------------------------------------------------
\renewcommand{\bibname}{\zihao{2}{\hspace{\fill}参\hspace{0.5em}考\hspace{0.5em}文\hspace{0.5em}献\hspace{\fill}}}

%---------------------------------------------------------------------
%	引用文献设置为上标
%---------------------------------------------------------------------
\makeatletter
\def\@cite#1#2{\textsuperscript{[{#1\if@tempswa , #2\fi}]}}
\makeatother

%---------------------------------------------------------------------
%	目录页设置
%---------------------------------------------------------------------
\titlecontents{chapter}[0em]{\songti\zihao{-4}}{\thecontentslabel\ }{}
{\hspace{.5em}\titlerule*[4pt]{$\cdot$}\contentspage}
\titlecontents{section}[2em]{\vspace{0.1\baselineskip}\songti\zihao{-4}}{\thecontentslabel\ }{}
{\hspace{.5em}\titlerule*[4pt]{$\cdot$}\contentspage}
\titlecontents{subsection}[4em]{\vspace{0.1\baselineskip}\songti\zihao{-4}}{\thecontentslabel\ }{}
{\hspace{.5em}\titlerule*[4pt]{$\cdot$}\contentspage}


\begin{document}
%---------------------------------------------------------------------
%	封面设置
%---------------------------------------------------------------------
\begin{titlepage}
	\begin{center}
		
    \includegraphics[width=0.4\textwidth]{figure//NJU.jpg}\\
    \vspace{10mm}
    
    \textbf{\zihao{1}\heiti{文献综述}}\\[0.8cm]
    
	\vspace{\fill}
	
\setlength{\extrarowheight}{3mm}
{\songti\zihao{3}	
\begin{tabular}{rl}
	
	{\makebox[4\ccwd][s]{项\qquad 目:}}& ~\kaishu 嵌入式系统的集群计算研究 \\ 

    {\makebox[4\ccwd][s]{成\qquad 员:}}& ~\kaishu 高博文~陈建~佘加辉~吴沛聪 \\ 
   
	{\makebox[4\ccwd][s]{指导教师:}} & ~\kaishu 方元\\ 

\end{tabular}
 }\\[2cm]
\vspace{\fill}
\zihao{4}
\today
	\end{center}	
\end{titlepage}

%---------------------------------------------------------------------
%  摘要页
%---------------------------------------------------------------------
\iffalse
\begin{abstract}
\begin{spacing}{1.5}
	{\zihao{-4}
	\\[0.5cm]
	\textbf{关键字}:\quad A \quad B \quad C \quad D
	}
\end{spacing}
\end{abstract}
\fi

%---------------------------------------------------------------------
%  目录页
%---------------------------------------------------------------------
\tableofcontents % 生成目录

%---------------------------------------------------------------------
%  Chapter: 文献综述
%---------------------------------------------------------------------
\chapter{文献综述}
\setcounter{page}{1}
\begin{spacing}{1.5}
\songti\zihao{-4}

\section{研究背景及意义}
随着计算机技术的不断发展、物联网应用的兴起和通信技术的革新,嵌入式设备作为可移动设备、物联网终端和广域网组网中的重要组成部分正发挥着越来越重要的作用,伴随着应用环境和应用需求的多样化与复杂化,嵌入式设备的计算能力需求正面临着越来越大的挑战。因为嵌入式系统计算能力的限制,在很多应用场景中它仍然扮演着数据收集、传感器控制、信号处理等并不需要太高计算能力的角色。

解决嵌入式系统计算能力缺陷的一个方法是将计算任务通过网络发送给计算中心,也就是大型计算机集群来完成。但是,这种方式对部署环境的通信能力和通信环境提出了较高要求,同时对部署环境的限制较大,在某种程度上还影响了终端用户的用户体验,由于嵌入式系统计算能力和通信能力的限制,有的设计因此无法被实现。李安的一个团队在一次测试中指出从一个Amazon的EC2计算实例到全球多个节点的平均回环时间达到了74ms,最高延迟甚至达到了500ms\cite{Satyanarayanan.{2017}},而且由于全球网络环境中某些特殊”局域网“的限制,网络连通性是没有稳定保障的。对移动设备的智能化的要求更是将提高嵌入式设备的计算能力置于不得不解决之地。提高嵌入式系统的计算性能可以通过改进芯片制作工艺,设计结构更为先进的嵌入式处理器来完成,例如2017年的高通基于第二代FinFTF工艺的ARMv8架构处理器骁龙835上集成了30亿个晶体管,比上一代骁龙820降低了20\%的功耗,提升了约100\%的计算能力\cite{ZhouPei.{2017}}。工艺上的改进的确可以提高处理器的计算性能,但是不可避免地会遇到价格过高限制用户群体以及灵活性不高的问题。因此在增强嵌入式处理器性能的同时,创建合理的嵌入式系统工作结构、改进嵌入式系统调度算法成为了尤为重要的问题。

构建一定规模的嵌入式集群就是一个不错的解决方案,它可以充分发挥嵌入式设备可扩展性与灵活性的优势,在降低整体功耗的同时解决单个核心计算能力不高的问题。同时因为RTOS、RT-Thread、Windows IoT等一系列嵌入式实时系统的存在,搭建一个嵌入式集群的困难程度已经不大。对嵌入式系统集群的进一步研究对提高嵌入式系统计算能力使其支持更为复杂、更加庞大的应用有着十分重要的意义。

\section{研究现状}
早在1994年,美国航空航天局就开始搭建Bewulf计算集群,它使用以太网实现计算机之间的通信与任务调度,实现了较高的性价比\cite{Gropp.{2002}}。近五年间,很多学者已经开始了对搭建嵌入式集群的尝试并对其进行性能测试。桑坦德工业大学的CJ Barrios等人在今年发表了一篇论文展示了他们对低功耗集群的能耗模型的研究\cite{Barrios.{2017}},而加拿大McMaster大学的N.Balakrishnan教授的团队已经使用搭载ARM Cortex-A9处理器的Pandaboard ES和搭载A11的Raspberry Pi 2搭建了一个嵌入式集群。他们的集群在并行计算的条件下实现了较高的能效和性能,但是随着集群节点数的增加集群性能的增长下降很快,同时增加节点导致的网络限制也制约了集群的进一步规模扩大和性能提升。之后的一些实验在此基础上通过提供更多的服务进行了进一步测试。Simon J Cox的团队则通过搭建了一个由64个Raspberry Pi的高性能嵌入式集群"Iridis-pi"进行了进一步的研究性实验\cite{Cox.{2014}}。他们在改善集群的扩展性问题上提出了使用SATA存储和PCIe总线的解决方案。在国内,湖南大学的黄一智提出了在集群上加入CPU、GPU异构节点来提高集群的性能以及算法的实时性的观点\cite{Huang.{2016}}。他使用Banana Pi搭建了一个16节点的集群并且使用Canny算法对他构建的集群效率进行了测试,但是他也没能在更大规模的应用上进行进一步研究。

纵观以上的实验及其总结我们不难发现对嵌入式集群的研究依然还停留在集群的构建和简单的能耗、计算速度测试上,没能在应用层面提出一个解决方案。同时在互相研究集群的构建时他们发现不同的处理器架构、不同的开发平台、甚至不同的开发板电源都会导致整体能耗和性能的不同,集群构建的可移植性和跨平台可用性并没有得到解决,面对着种类繁多的嵌入式开发平台,解决可移植性和跨平台搭建集群以及共用接口协议是十分关键并且需要解决的问题。同时在Simon的研究中尽管使用了新的存储和总线方案,他们也没能在很大程度上提高集群的可扩展性,也就是说没能发挥整个集群可能发挥的最大性能。同时对嵌入式系统的任务调度的研究也处于较为初级的阶段,而任务调度是集群与分布式计算中一个重要的环节。

\section{研究方向}
基于以上对已经进行的实验和他们存在的问题和不足的分析,进一步的研究我们将会从下面几个部分进行。

首先是构建嵌入式集群。大部分已经有过的研究都是用了诸如Raspiberry Pi、Pandaboard等开源硬件平台,我们在重复他们的实验的基础上针对实验室中的九鼎创展x210ii开发平台搭建集群。通过网络或者PCIe作为传输途径,测试传输效率。

然后需要进行的是任务调度和并行算法的开发。通过在集群中运行分布式计算程序来测试嵌入式集群的运行效率和性能指标。通过增加集群节点进行进一步测试发现限制增加节点集群性能增加的原因,给出可能的改进措施并且进行测试。

下一步是创建API接口来实现集群应用的多点部署,通过提供公用的API接口实现算法在不同集群中的复用。

\end{spacing}

%---------------------------------------------------------------------
%  参考文献设置
%---------------------------------------------------------------------
\addcontentsline{toc}{chapter}{参考文献}

\begin{thebibliography}{99}
\songti \zihao{-4} 
\bibitem{Satyanarayanan.{2017}}
Satyanarayanan M. The Emergence of Edge Computing[J]. Computer, 2017, 50(1):30-39.

\bibitem{ZhouPei.{2017}}
周培. 骁龙835性能简析, 10纳米工艺再进一步[J]. 计算机与网络, 2017, 43(2):48-49.

\bibitem{Gropp.{2002}}
Gropp W, Lusk E L, Sterling T. Beowulf cluster computing with Linux[M]. Beowulf cluster computing with Linux, MIT Press, 2002.

\bibitem{Barrios.{2017}}
Barrios C J, Emeras J, Pecero J E. Energy model for low-power cluster[C].  Ieee/acm International Symposium on Cluster, Cloud and Grid Computing. IEEE Press, 2017:1009-1016.

\bibitem{Cox.{2014}}
Cox S J, Cox J T, Boardman R P, et al. Iridis-pi: a low-cost, compact demonstration cluster[J]. Cluster Computing, 2014, 17(2):349-358.

\bibitem{Huang.{2016}}
黄一智. 高性能并行分布式嵌入式集群构建与应用研究[D]. 湖南大学, 2016.

\end{thebibliography}
%---------------------------------------------------------------------
%  附录设置
%---------------------------------------------------------------------
\titleformat{\chapter}{\heiti\Large}{附录~\Alph{chapter}}{11pt}{\Large}
\titlespacing{\chapter}{0pt}{*-4}{*4}

\lstset{breaklines}                %自动将长的代码行换行排版
\lstset{extendedchars=false}
\lstset{language=Matlab}

\renewcommand{\thechapter}{附录\Alph{chapter}.} 
\appendix
\begin{appendix}
\begin{spacing}{1.5}

\end{spacing}
\end{appendix}
		

\end{document}